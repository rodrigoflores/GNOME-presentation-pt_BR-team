\documentclass{beamer}
\usepackage[brazil]{babel}
%\usepackage[latin1]{inputenc}
\usepackage[utf8x]{inputenc} 
%\usepackage[all]{xy}
\PrerenderUnicode{ç}

\setbeamercovered{transparent=5}

\usetheme{Antibes}

\title{Projeto de Tradução do GNOME para o Português do Brasil: presente e futuro}

\author{Rodrigo L. M. Flores \\ \url{mail@rodrigoflores.org}}

\logo{\includegraphics[width=1cm]{gnome-logo}}

\institute{GNOME Brasil}
\begin{document}

\date{\today}

\frame{\titlepage}

\frame{\tableofcontents}

\section{Passado}

\begin{frame}
    \frametitle{E onde está o passado ?}
    Informações incertas demais para se falar ! Entrei em dezembro de 2007, então muita gente eu não tive
    o prazer de trabalhar junto. 
\end{frame}

\begin{frame}
    \frametitle{Mas sem esquecer de citar}
    \begin{columns}[c]
      \column{1.5in}
        \begin{itemize}
          \item Raphael Higino
          \item Gustavo Noronha
          \item Evandro Giovanini
          \item Alexandre Folle de Menezes
          \item Maurício Collares
          \item Guilherme Pastore
        \end{itemize}
       \column{1.5in}
        \begin{itemize}
          \item Estevão Procópio
          \item Sandro Nunes Henrique 
          \item Luiz Armesto
          \item Pedro de Medeiros
          \item Igor Soares
          \item Hugo Dória
          \item Raúl Pereira
        \end{itemize}
    \end{columns}
\end{frame}

\section{Um pouco sobre nós}

\begin{frame}
    \frametitle{Missão}
    Fazer com que usuários do GNOME possam utilizar um sistema 
    totalmente traduzido para nosso idioma, cumprindo com padrões altos de qualidade 
\end{frame}

\begin{frame}
    \frametitle[Para o GNOME 2.28]{Nosso status}    
    \begin{itemize}[<+->]
        \item \color[rgb]{0,1,0}   Somos citados nas notas de lançamento desde o GNOME 2.4 (Setembro de 2003)  
    
        \item \color[rgb]{0,1,0}  100\% da Interface do Oficial 
        \item \color[rgb]{0,1,0}  87\% da Interface do Extra
        \item \color[rgb]{0,1,0}  82\% da Interface de Infraestrutura
        \item \color[rgb]{1,0,0} 36\% de Documentação do Oficial
        \item \color[rgb]{1,0,0} 9\%  de Documentação do Extra
        \item \color[rgb]{1,0,0} 31\% de Documentação de Infraestrutura
    \end{itemize}

\end{frame}

\section{Como funcionávamos ?}

\begin{frame}
    \frametitle{Nosso fluxo de trabalho antigo}
    \begin{figure}[ht]
        \includegraphics{fluxo_antigo.eps}     
    \end{figure}
\end{frame}

\begin{frame}
    \frametitle{Fluxo de tradução antigo}
    \begin{enumerate}[<+->]
        \item Tradutor escolhe um módulo não reservado no Wiki
        \item Tradutor baixa o arquivo po no Damned Lies
        \item Tradutor traduz o arquivo po
        \item Tradutor envia o arquivo po ao Bugzilla
        \item Commiter revisa a tradução 
        \item Commiter envia ao Bugzilla e marca o ``bug'' como resolvido
        \item Commiter faz o commit no SVN
    \end{enumerate}
\end{frame}

\begin{frame}
    \frametitle{Falhas?}
    \begin{itemize}[<+->]
        \item Processo muito dependente do commiter
        \item Reservas no Wiki as vezes eram esquecidas e aí módulos estavam ``pseudo-reservados''
        \item Processo muito complicado
    \end{itemize}
\end{frame}

\section{Uma nova era...}

\begin{frame}
    \frametitle{O que tinhamos?}
    \begin{itemize}[<+->]
        \item Tradutores experientes;
        \item Vertimus;
    \end{itemize}
\end{frame}

\begin{frame}
    \frametitle{Nova proposta}  
    Em Abril de 2008, o Leonardo Fontenelle sugeriu uma nova proposta de organização da equipe:
    \begin{itemize}[<+->]
        \item Criação de uma nova classe: o revisor
        \item Promoção de um tradutor para revisor dependentemente principalmente dos outros revisores/committers
        \item Confiança no revisor
    \end{itemize}
\end{frame}

\begin{frame}
    \frametitle{Novo fluxo de trabalho}
    \begin{figure}[ht]
        \includegraphics{fluxo_novo.eps}     
    \end{figure}
\end{frame}

\begin{frame}
    \frametitle[Vertimus]{Novo fluxo de tradução}
    \begin{enumerate}[<+->]
        \item Tradutor reserva e baixa a tradução no Vertimus
        \item Tradutor sobe o arquivo traduzido ao Vertimus
        \item Revisor reserva a tradução
        \item Revisor sobe o arquivo revisado ao Vertimus
        \item Commiter baixa o arquivo do Vertimus
        \item Commiter avisa da submissão
    \end{enumerate}
\end{frame}

\begin{frame}
    \frametitle{Resolve os falhas antigas ?}
    \begin{itemize}[<+->]
        \item Ainda somos dependentes dos Commiters, mas só para enviar tradução
        \item Traduções acumulam menos
        \item Reserva do módulo é redefinida a cada commit
        \item Processo bem mais simples
    \end{itemize}
\end{frame}

\begin{frame}
  \frametitle{Falhas novas}
\end{frame}

\begin{frame}
    \frametitle{E como tradutores viram revisores?}
    Através de um pedido na lista.

    Requisitos:
    \begin{itemize}[<+->]
        \item Fazer parte da equipe há, pelo menos, 6 meses
        \item Ter traduzido algum pacote ``não-trivial''
        \item Ser aprovado por alguém que revisou algo recentemente
        \item Ser aprovado por pelo menos 2 commiters (e o coordenador pode ser um desses)
        \item Ser aprovado pelo coordenador
    \end{itemize}
\end{frame}

\begin{frame}
    \frametitle{E como tradutores viram revisores?}
    Através de um pedido na lista.
    Requisitos:
    \begin{itemize}[<+->]
        \item Fazer parte da equipe há, pelo menos, 6 meses
        \item Ter traduzido algum pacote ``não-trivial''
        \item Ser aprovado por alguém que revisou algo recentemente
        \item Ser aprovado por pelo menos 2 commiters (e o coordenador pode ser um desses)
        \item Ser aprovado pelo coordenador
    \end{itemize}
\end{frame}



\section{Dando nome aos bois}

\begin{frame}
    \frametitle{``Committers''}
    \begin{itemize}
        \item Leonardo Fontenelle 
        \item John Wendell
        \item Og Maciel
    \end{itemize}
\end{frame}

\begin{frame}[fragile]
    \frametitle{Revisores}
    \begin{itemize}
        \item Todos os committers
        \item Vladimir Melo 
        \item Fábio Nogueira aka \verb#ubuntuser#
        \item Djavan Fagundes
        \item Fabrício Godoy
        \item Henrique Machado aka \verb#zehrique#    
    \end{itemize}
\end{frame}

\begin{frame}[fragile]
    \frametitle{Tradutores}
    \begin{columns}[c]
      \column{1.5in}

      \begin{itemize}
        \item Todos os revisores;
        \item Enrico Nicoletto
        \item Krix Apolinário
        \item André Gondim
        \item César Veiga
        \item Carlos Pereira
      \end{itemize}

      \column{1.5in}
      \begin{itemize}
        \item Daniel Koda
        \item Tiago Casal
        \item Rodolfo Gomes
        \item Jader Henrique 
        \item Wancharle Quirino
      \end{itemize}
    \end{columns}

\end{frame}

\section{Futuros planos}

\begin{frame}
    \frametitle{Iniciantes?}
    Nossa equipe tem um forte incentivo à entrada de iniciantes. Para isso poderia haver um ``plano de iniciação''.
    \begin{enumerate}
        \item Corrigir e traduzir pequenas coisas, normalmente adições de strings ou corrigir erros. 
        \item Traduzir interface mais simples (faltando menos de 20 strings)
        \item Traduzir interface mais complexos 
        \item Traduzir interfaces do zero
        \item Traduzir documentações conhecidas
    \end{enumerate}
\end{frame}

\begin{frame}
    \frametitle{Reforma ortográfica}

    Adequar as traduções do GNOME à reforma ortográfica
    \begin{itemize}[<+->]
        \item Manualmente ? 
        \item Via algum script ?
        \item Rodar um corretor ortográfico em todas as mensagens ?  
    \end{itemize}
\end{frame}

\begin{frame}
    \frametitle{Bolar alguma regra de crédito de autoria}
    Se um tradutor corrige algo do tamanho de um typo, ele deveria ser considerado o último tradutor e também receber crédito pela tradução ? 
\end{frame}

\begin{frame}
    \frametitle{Tradução incremental de documentações grandes}
    Algumas documentações têm centenas de mensagens, boa parte delas do tamanho de parágrafos. 
    Será que um envio incremental não ajudaria a melhorar a qualidade delas?
\end{frame}

\begin{frame}
    \frametitle{Revisão obrigatória para traduções grandes}
    Quanto maior a tradução, maior a chance de pequenos erros persistirem. 
    Não seria bom as traduções grandes serem revisadas independentementes do tradutor?
\end{frame}

\begin{frame}
    \frametitle{Tradução via web ou via arquivos po?}
    Tradução via web já existe no universo Ubuntu e possui algumas vantagens. Será que seria bom
    fazermos isso no GNOME ?
\end{frame}

\begin{frame}
    \begin{block}{Contato}     
    \begin{itemize}
            \centering
            \item[E-mail] mail@rodrigoflores.org 
            \item[XMPP]  im@rodrigoflores.org        
            \item[Site]  \url{http://rodrigoflores.org}
            \item[Blog]  \url{http://blog.rodrigoflores.org}        
            \item[Twitter] rlmflores 
            \item[Identi.ca] rodrigoflores        
            \item[Jaiku] flores        
        \end{itemize}
    \end{block}
\end{frame}


\end{document}


% vim:set ts=2 expandtab:
